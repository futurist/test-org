% Created 2016-05-05 Thu 13:00
\documentclass[11pt]{article}
\usepackage[utf8]{inputenc}
\usepackage[T1]{fontenc}
\usepackage{fixltx2e}
\usepackage{graphicx}
\usepackage{grffile}
\usepackage{longtable}
\usepackage{wrapfig}
\usepackage{rotating}
\usepackage[normalem]{ulem}
\usepackage{amsmath}
\usepackage{textcomp}
\usepackage{amssymb}
\usepackage{capt-of}
\usepackage{hyperref}
\setcounter{secnumdepth}{3}
\author{5分钟学院}
\date{\textit{<2016-05-05 Thu>}}
\title{Metalsmith使用方法教程}
\hypersetup{
 pdfauthor={5分钟学院},
 pdftitle={Metalsmith使用方法教程},
 pdfkeywords={},
 pdfsubject={},
 pdfcreator={Emacs 24.5.1 (Org mode 8.3.4)}, 
 pdflang={Zh-Cn}}
\begin{document}

\maketitle
\setcounter{tocdepth}{2}
\tableofcontents


\section{Metalsmith简介,为何?}
\label{sec:orgheadline3}
\href{http://metalsmith.io/}{Metalsmith} 是使用NodeJS进行文件处理的库,一般用于静态站生成。

\subsection{基本流程}
\label{sec:orgheadline1}

\begin{enumerate}
\item \texttt{.src} 读取 \textbf{源文件} ( .src(\emph{dirname}) ), \emph{dirname} 默认 \texttt{src/} 目录

\begin{itemize}
\item 每个源文件由两部分组成:
\begin{enumerate}
\item \href{https://middlemanapp.com/basics/frontmatter}{YAML-Frontmatter} 头部(YAML header)
\item 内容区(contents)
\end{enumerate}
\item 解析源文件生成JS Object, 送入下一层(\href{https://github.com/segmentio/ware}{middleware})
\end{itemize}

\item \texttt{.use} 使用一步或多步插件对上一步的JS Object进行处理。送入下一层
\item \texttt{.build} 将最终处理的contents写入相应文件,默认存放 \texttt{build/} 目录
\end{enumerate}

\subsection{常用插件}
\label{sec:orgheadline2}

\begin{itemize}
\item \href{https://github.com/superwolff/metalsmith-layouts}{metalsmith-layouts}
\item \href{https://github.com/segmentio/metalsmith-markdown}{metalsmith-markdown}
\item \href{https://github.com/segmentio/metalsmith-permalinks}{metalsmith-permalinks}
\item \href{https://github.com/segmentio/metalsmith-collections}{metalsmith-collections}
\end{itemize}

\section{优点}
\label{sec:orgheadline4}

\begin{itemize}
\item 配置灵活简单,容易上手
\item 插件容易写,可与Nodejs各种类库无缝接合
\item 生成的静态站无服务器运算,最小化服务器负担
\end{itemize}

\section{应用}
\label{sec:orgheadline5}

\begin{itemize}
\item 博客(blog)
\item 小型CMS站点(公司或个人)
\item 对响应时间有要求的站点(静态无服务器运算)
\end{itemize}

\section{搭建博客过程}
\label{sec:orgheadline6}
\end{document}